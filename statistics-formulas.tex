\documentclass[landscape]{article}

\usepackage{amsmath}
\usepackage{booktabs}
\usepackage{pgfplots}
\usepackage{hyperref}

\title{Statistic Formulas}
\author{RobinXSI}
\date{Juli 2015}
\begin{document}
   \maketitle
   	\section{Introduction}

   		Wichtig beim aufstellen von Hypothesen ist, dass die Auswertung eine statistische Signifikanz haben.

   		Statistical significance means:
		\begin{itemize}
			\item rected the null hypothesis
			\item results are not likely due to chance (sampling error)
		\end{itemize}

		Als Endresultat der Statistik sollten folgendes Dokument erstellt werden können:
		\begin{itemize}
			\item Desciriptive Statistics
				\begin{itemize}
					\item M - Mean
					\item Sd - Standard Deviation
				\end{itemize}
			\item Inferential Statistics
				\begin{itemize}
					\item Hypothesis Test \(\alpha\)
					\begin{itemize}
						\item kind of test, bsp: one-sample t-test
						\item test statistic, bsp: t-value
						\item df - degrees of freedom
						\item p-value
						\item direction of test, bsp: one-tail-test or two-tail-test
					\end{itemize}
					\item APA style
						  \\\(t(df) = X.XX, p = X.XX,\) direction
						  \\bsp: \(t(24) = -2.50, p < .05,\) one-tailed
					\item Confidence intervals
					\begin{itemize}
						\item Confidence leve, bsp: 95\%
						\item Lower Limit
						\item Upper Limit
						\item CI on what?
					\end{itemize}
					\item APA style - CIs
						  \\Confidence interval on the mean difference;
						  \\95\% CI = (4 to 6)
					\item Effect size measures
						\begin{itemize}
							\item \(d\)
							\item \(r^2\)
						\end{itemize}
					\item APA style - CIs
						  \\\(d = X.XX\)
						  \\\(r^2 = .XX\)

				\end{itemize}
		\end{itemize}


	\section{Statistical Formulas}
		\pgfmathdeclarefunction{gauss}{2}{%
		  \pgfmathparse{1/(#2*sqrt(2*pi))*exp(-((x-#1)^2)/(2*#2^2))}%
		}

		\begin{tikzpicture}
			\begin{axis}[
			  no markers, domain=0:10, samples=100,
			  axis lines*=left, xlabel=$x$, ylabel=$y$,
			  every axis y label/.style={at=(current axis.above origin),anchor=south},
			  every axis x label/.style={at=(current axis.right of origin),anchor=west},
			  height=5cm, width=15cm,
			  xtick={4}, ytick=\empty,
			  enlargelimits=false, clip=false, axis on top,
			  grid = major
			  ]
			  \addplot [fill=cyan!20, draw=none, domain=0:2.4] {gauss(4,1)} \closedcycle;
			  \addplot [fill=cyan!20, draw=none, domain=5.6:10] {gauss(4,1)} \closedcycle;
			  \addplot [very thick,cyan!50!black] {gauss(4,1)};
			 


			\draw [yshift=-0.6cm, latex-latex](axis cs:4,0) -- node [fill=white] {$1.96\sigma$} (axis cs:5.6,0);
			\end{axis}

		\end{tikzpicture}

		\subsection{Empirical Mean}
		\(\gamma = \frac{\sum{x}}{N}\)

		Properties (for independent random variables X and Y):
		\begin{enumerate}
			\item \(Mean(X + Y) = Mean(X) + Mean(Y)\)
			\item \(Mean(X \times Y) = Mean(X) \times Mean(Y)\)
		\end{enumerate}

		\subsection{Variance}
		\(\sigma^2 = Var(X)\newline
			\sigma^2 = \frac{\sum{(x_i - \gamma)^2}}{N}\newline
			\sigma^2 = \frac{\sum{X_i^2}}{N} - \frac{(\sum{X_i})^2}{N^2}\newline
			\sigma^2 = \frac{\sum{(x_i - \gamma)^2}}{N}\)\newline
		
		\subsection{Standard Deviation}
		\(\sigma = \sqrt{Var(X)}\newline
			\sigma = \sqrt{\frac{\sum{(X_i - \gamma)^2}}{N}}\)\newline

		\subsection{Standard Error}
		\(Sd = \frac{\sigma}{N}\)

		\subsection{Z-Score}
		Z-Score \(= \frac{x - \bar{x}}{Sd}\)

		\subsection{Standard Normal Distribution}
		\(\frac{1}{\sqrt{2 \cdot \pi \cdot \sigma^2}} \cdot e^{[-\frac{1}{2} \cdot \frac{(x-\gamma)^2}{\sigma^2}]}\)

		\subsection{Confidence Interval}
		\(CI = 1.96 \cdot \sqrt{\frac{p(1 - p)}{N}}\newline
			General Form:\newline
			Size of CI = a \cdot \sqrt{\frac{\sigma^2}{N}}\newline
			\frac{\sum{X_i \pm a \cdot \sqrt{\frac{\sigma^2}{N}}}}{N}\)

			\textbf{Note}

			\begin{enumerate}
			\item $a = 1.96 for N \geq 30$
			\item $a$ is the t-value computed for $(N - 1)$ degrees of freedom and confidence level $p$.
			\end{enumerate}

		\subsection{Additional Formulas}
		\begin{tabular}{lllll}
			\hline
			Name & Population Symbol & Sample Symbol & Sample Calculation & Beschreibung \\ \hline
			Mean &  & $\bar{x}$ & $\bar{x} = \frac{\sum{x}}{N}$ & Durchschnitt aller Daten \\
			Variance & $\sigma_x^2$ & $s_x^2$ & $s_x^2 = \frac{\sum{(x - \bar{x})^2}}{N - 1}$ & Abweichung \\
			Standard Dev & $\sigma_x$ & $s_x$ & $s_x = \sqrt{s_x^2}$ &  \\
			Covariance & $\sigma_xy$ & $s_xy$ & $s_xy = \frac{\sum{(x - \bar{x})(y - \bar{y})}}{N - 1}$ &  \\
			Correlation & $\rho_xy$ & $r_xy$ & \begin{tabular}[c]{@{}l@{}}$r_xy = \frac{s_xy}{s_x s_y}$\\ $r_xy = \frac{\sum{(z_x z_y)}}{N - 1}$\end{tabular} &  \\ \hline
			z-score & $z_x$ & $z_x$ & $z_x = \frac{x-\bar{x}}{s_x}; \bar{z} = 0; s_x^2 = 1$ &  \\ \hline
		\end{tabular}

	\subsection{T-Test}
		\[
		t = \frac{\bar{x_D} - 0}{s_D / \sqrt{n}}
		\]

	\subsection{Hypothese}
		``US families spent an average of \$151 per week on food''
		\\Beispiel Null Hypothese: ``the program did not change the cost of food''
		\\Beispiel Alternative Hypothese: ``the program reduced the cost of food''
		\\
		\\\(null \rightarrow H_0:\mu_program >= 151\)
		\\\(alt \rightarrow H_A:\mu_program < 151\)


	\section{10b: t-Test, Part 2}

		Difference Measures
		\begin{itemize}
			\item meandifference
			\item standardized differences
			\item chohen's d
		\end{itemize}	

		Correlation measures
		\begin{itemize}
			\item \(r^2\) - proportion (\%) of variation in one variable that is related to (``explained by'') another variable.
		\end{itemize}

		\subsection{Cohen's d}
			Standardized mean difference 
			\\\(d = \frac{\bar{X} - \mu}{Sd}\)

		\subsection{Correlation measures}
			\(r^2 \Rightarrow\) bestimmt die Stärke des Zusammenhangs zwischen zwei Variablen als Proportion (\%)
			\\Wert zwischen 0.00 und 1.00
			\\\(r^2 = \frac{t^2}{t^2 + df}\)
			\\\(df = \) degrees of freedom


		\subsection{Important formulas}
			\(\mu \Rightarrow\) Population Mean
			\\\(n \Rightarrow\) Sample Size
			\\\(\bar{X} \Rightarrow\) Sample Mean
			\\\(s_D \Rightarrow\) Standard Deviation for sample
			\\\(\alpha \Rightarrow\) tail probability on t-table
			\\\(t_{critical} \Rightarrow\) from \href{https://s3.amazonaws.com/udacity-hosted-downloads/t-table.jpg}{t-table}
			\\\(df = n - 1 \Rightarrow\) Degrees of Freedom
			\\\(SEM = \frac{s_D}{\sqrt{n}} \Rightarrow\) Standard Error of the Mean
			\\\(t = \frac{\bar{X} - \mu}{SEM} \Rightarrow\) One Sample t-Test
			\\margin of error \(= (t^{critical} \times SEM)\)
			\\\(CI = \bar{X} \pm \) margin of error
			\\\(d = \frac{\bar{X}-\mu}{s_D} \Rightarrow\) Cohen's d
			\\\(r^2 = \frac{t^2}{t^2 + df}\)
	

\end{document}